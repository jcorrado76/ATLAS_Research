\chapter{Introduction}
\section{The Large Hadron Collider}
The Large Hadron Collider is the most powerful particle accelerator in the world, located in CERN on the France-Swiss border.
The circumference of the LHC is 27km, and is located 100m underground. 
The maximum energy possible to accelerate the particles to in the LHC is directly dependent on the size of the LHC and the strength of the magnets used to accelerate the particles. 
In order to achieve the design energy of 7 TeV per proton, it was necessary to have a circumference of 27km, and to use some of the most powerful dipoles and radiofrequency cavities in existence. 
Inside the pipes where the protons travel, a very strong vacuum is required, and so the pressure in some parts is over $10^{-9}Pa$. 
The beams are made up of cylinder-like bunches. 
Using these bunches, the expected number of collisions is $10^{34}$ per $cm^2$ per second. 
The time between bunches is about $25ns$, so in the entire circumference of the LHC, there are about $3550$ bunches.
\section{ATLAS Experiment}
\textbf{ATLAS (A Toroidal LHC ApparatuS)} is one of seven particle detector experiments constructed at the Large Hadron Collider, a particle accelerator at CERN. 
When the LHC runs at full energy and intensity, about 600 million proton-proton collisions take place every second inside the ATLAS detector. 
The amount of data collected for each event is around 1MB, which means that there is approximately:
$$10^9Hz(1Mb)=1Pb/s$$
of data produced. 
Because this is significantly larger than any practical system can handle, there are \textbf{triggers} that are designed to reject uninteresting events and keep the interesting ones. 
For ATLAS, the trigger system is designed to collect about 200 events per second. 
This means that ATLAS collects about 4 petabytes of data per year. 
There are $10^{11}$ protons in a bunch. 
The proton-proton interaction cross section is approximately $100mB$.


\section{Trigger System}
In particle physics, a trigger is a system that uses simple criteria to rapidly decide which events in a particle detector to keep when only a small fraction of the total can be recorded.
The trigger system is necessary because of limitations in terms of data storage capacity and rates. 
In general, the experiments typically search for ``interesting'' events (decays of rare particles) that occur at relatively low rates, so we need to have trigger systems that identify events that should be recorded for later analysis. 
The Large Hadron Collider has an event rate of approximately 1 GHz. 
The triggers are divided into levels so that each level selects the data that becomes an input for the next level, which has more time available and more information to make better decisions.
There is the \textbf{Level-1 (L1)} system, which is based on custom electronics, and the \textbf{High Level Trigger (HLT)} system, that relies on commercial processors. 
The L1 system uses only coarsely segmented data from the calorimeter and muon detectors, while holding all the high-resolution data in pipeline memories in the electronics.
\section{Missing Transverse Momentum}
The transverse momentum is defined to be the momentum in the transverse plane to the beam axis. 
Because the protons in the beam pipe collide approximately head-on in opposite directions, we expect the produced particles to have approximately zero momentum. 
As a result, we know we must conserve transverse momentum. 
The total transverse momentum is defined to be the vector sum of the transverse momenta of the particles produced in the beam pipe. 
Missing transverse momentum is defined as the vector sum of the transverse momenta of all invisible particles.
The missing transverse momentum can be understood as the amount by which produced particles are failing to obey conservation of the three-momentum.
Therefore, by only using muon-triggered events that passed a W transverse mass cut enriches the W and therefore real MET content.
