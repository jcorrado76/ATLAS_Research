\section{Bisection}
In this assignment, we wanted to see if we could obtain an increase in efficiency by combined some uncorrelated algorithms together, subject to the constraint of the trigger rate. 
The way I did this was to try all combinations of algorithms, impose the constraint that the algorithms keep the same fraction individually, and then I performed bisection along that line in parameter space to find the value closest to the trigger rate. 
There are two level curves of interest in this project. 
The first one is what I call the production possibility frontier, to borrow a term from economics. 
This curve represents the solution space to the set of pairs of thresholds one could use for the pair of algorithms that satisfies the constraint of the trigger rate.
The second curve that is illustrating to look at is the curve showing the pairs of points on the curve where we constrain the two algorithms to individually keep the same fraction (the line $y=x$ in the parameter space). 
We want to find the pair of thresholds for the algorithms that when we use both of the algorithms, such that they individually keep the same fraction of events, keep the trigger rate [fraction of events]. 
The level curve describing the set of pairs of thresholds such that the trigger rate constraint is satisfied is given by the constraint:
$$f(\tau_{\alpha},\tau_{\beta})=C$$
for some $C$. Here, $f$ is the function representing the fraction of events kept when the algorithms are used together at the same time. 
We expect $f$ to be a monotonic decreasing function in the thresholds, as increasing the threshold would cause fewer events to be kept by the algorithm.
In order to compute $C$, we used the fraction of passnoalg data that passed an L1 MET cut of $50$ GeV and a CELL MET cut of $100$ GeV. 
For our analysis, $C$ turned out to be $0.0059$.
So we needed to solve the equation $f(\tau_{\alpha},\tau_{\beta})=0.0059$. 
However, because the parameter space is two-dimensional, and the evaluation of $f$ takes a long time (fraction of events kept by both algorithms, and by each one individually), we introduced the constraint that the two individual fractions kept needed to be the same.
This turned our problem into a one dimensional one, as we were solving for the intersection of the production possibilities frontier curve and the constraint curve on the individual fractions. 
Then, we were able to solve this one dimensional problem by using the root-finding bisection algorithm on each of the pairs of high level trigger algorithms.  
\subsection{Transverse Mass Cut}
In addition to the cuts on the various algorithms, we also needed to introduce a cut on the transverse mass that is detected to ensure we only keep events with a transverse mass close to that of the W boson ($80.379\pm 0.012 GeV/c^2$). 
We compute the transverse mass using:
$$m_{T}=\sqrt{2P_{\mu}P_{\nu}(1+\cos{(\phi)})}$$
In addition to the aforementioned cuts, we also added a cut on this quantity for the range $40 \leq m_{T} \leq 100$. 
\section{Results}
We found that we were able to achieve an increase in the overall efficiency for some of the pairs of algorithms considered. 

