\section{Bisection}
In this assignment, we wanted to see if we could obtain an increase in efficiency by combined some uncorrelated algorithms together, subject to the constraint of the trigger rate. 
The way I did this was to try all combinations of algorithms, impose the constraint that the algorithms keep the same fraction individually, and then I performed bisection along that line in parameter space to find the value closest to the trigger rate. 
There are two level curves of interest in this project. 
The first one is what I call the production possibility frontier, to borrow a term from economics. 
This curve represents the solution space to the set of pairs of thresholds one could use for the pair of algorithms that satisfies the constraint of the trigger rate.
The second curve that is illustrating to look at is the curve showing the pairs of points on the curve where we constrain the two algorithms to individually keep the same fraction (the line $y=x$ in the parameter space). 
